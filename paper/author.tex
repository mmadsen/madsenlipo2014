%%%%%%%%%%%%%%%%%%%% author.tex %%%%%%%%%%%%%%%%%%%%%%%%%%%%%%%%%%%
%
% sample root file for your "contribution" to a contributed volume
%
% Use this file as a template for your own input.
%
%%%%%%%%%%%%%%%% Springer %%%%%%%%%%%%%%%%%%%%%%%%%%%%%%%%%%


% RECOMMENDED %%%%%%%%%%%%%%%%%%%%%%%%%%%%%%%%%%%%%%%%%%%%%%%%%%%
\documentclass[graybox,natbib]{svmult}

% choose options for [] as required from the list
% in the Reference Guide
%% The amssymb package provides various useful mathematical symbols
\usepackage{amssymb,amsmath}

\usepackage{mathptmx}       % selects Times Roman as basic font
\usepackage{helvet}         % selects Helvetica as sans-serif font
\usepackage{courier}        % selects Courier as typewriter font
\usepackage{type1cm}        % activate if the above 3 fonts are
                            % not available on your system
%
\usepackage{makeidx}         % allows index generation
\usepackage{graphicx}        % standard LaTeX graphics tool
                             % when including figure files
\usepackage{multicol}        % used for the two-column index
\usepackage[bottom]{footmisc}% places footnotes at page bottom

\usepackage{lipsum}
\usepackage{url}
\usepackage{diss-macros}
\usepackage[section,ruled]{algorithm}
\usepackage{algorithmic}
\usepackage{boxedminipage}
\usepackage[xetex,bookmarks=true,linkcolor=blue,hyperfootnotes=false,breaklinks=true,citecolor=blue,colorlinks=true]{hyperref}
\usepackage{sistyle}
\SIthousandsep{,}

% see the list of further useful packages
% in the Reference Guide

\makeindex             % used for the subject index
                       % please use the style svind.ist with
                       % your makeindex program

%%%%%%%%%%%%%%%%%%%%%%%%%%%%%%%%%%%%%%%%%%%%%%%%%%%%%%%%%%%%%%%%%%%%%%%%%%%%%%%%%%%%%%%%%

\begin{document}

\title*{Behavioral Modernity and the Cultural Transmission of Structured Information: The Semantic Axelrod Model}
\titlerunning{Cultural Transmission of Structured Information}
% Use \titlerunning{Short Title} for an abbreviated version of
% your contribution title if the original one is too long
\author{Mark E. Madsen and Carl P. Lipo}
% Use \authorrunning{Short Title} for an abbreviated version of
% your contribution title if the original one is too long
\institute{Mark E. Madsen \at Dept. of Anthropology, University of Washington, Box 353100, Seattle, WA 98195 \email{mark@madsenlab.org}
\and Carl P. Lipo \at Department of Anthropology and IIRMES, California State University at Long Beach, 1250 Bellflower Blvd, Long Beach, CA  90840 \email{Carl.Lipo@csulb.edu}}
%
% Use the package "url.sty" to avoid
% problems with special characters
% used in your e-mail or web address
%
\maketitle

\abstract*{Cultural transmission models are coming to the fore in explaining increases in the Paleolithic toolkit richness and diversity. Analyses suggest that diversity increased due to relaxation of conformism, due to the effects of demographic expansion on cultural diversity, and the effects of extinction and recolonization in metapopulations. During the Paleolithic, however, technologies increase not only in terms of diversity but also in their complexity and interdependence. As \citet{Mesoudi2008a} have shown selection broadly favors social learning that is hierarchical and structured, rather than information which is piecemeal and independent. The addition of structured information acquisition potentially explains how the complexity of technology changes along with diversity. Here, we introduce a variant of Axelrod's model of cultural differentiation, modified such that homophily and conformism refers to the content or “semantics” of traits, instead of simply their frequencies. We examine the conditions under which structured suites of traits develop and differentiate in the model, which can represent the chains of prerequisites, “background” information, and local specializations characteristic of real technology traditions. Our results point to ways in which we can build more comprehensive explanations of the archaeological record of the Paleolithic as well as other cases of technological change.}

\abstract{Cultural transmission models are coming to the fore in explaining increases in the Paleolithic toolkit richness and diversity. Analyses suggest that diversity increased due to relaxation of conformism, due to the effects of demographic expansion on cultural diversity, and the effects of extinction and recolonization in metapopulations. During the Paleolithic, however, technologies increase not only in terms of diversity but also in their complexity and interdependence. As \citet{Mesoudi2008a} have shown selection broadly favors social learning that is hierarchical and structured, rather than information which is piecemeal and independent. The addition of structured information acquisition potentially explains how the complexity of technology changes along with diversity. Here, we introduce a variant of Axelrod’s model of cultural differentiation, modified such that homophily and conformism refers to the content or “semantics” of traits, instead of simply their frequencies. We examine the conditions under which structured suites of traits develop and differentiate in the model, which can represent the chains of prerequisites, “background” information, and local specializations characteristic of real technology traditions. Our results point to ways in which we can build more comprehensive explanations of the archaeological record of the Paleolithic as well as other cases of technological change.}


$body$


%% References with bibTeX database:

\bibliographystyle{model2-names}
\bibliography{$biblio-files$}

\end{document}
